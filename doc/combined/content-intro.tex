\part{Einführung}
\section{Motivation}
Den Himmel überzeugend darzustellen ist offensichtlich ein wichtiger Teil von AAA-Games bis hin zu Simulationen. Wenn es
Regnet, der Himmel aber strahlend blau und die Wolken unpassend weiß sind, ist der Realismus jeder Szene dahin. Mit dem
speziellen Ziel Echtzeitrendering implementieren wir hier in unserem eigenen Vulkan-basierten Renderer ein Modul das
sowohl die Atmosphäre als auch Wolken physikalisch basiert darstellt. Das gesamte System benötigt nur einen Renderpass,
was (in vielen Fällen) mobilen, Tile Based Renderern erlaubt eine hohe Effizienz zu erreichen (hinsichtlich des
wachsendes VR/AR-Marktes eine erwähnenswerte Eigenschaft).

Neben der physikalischen Korrektheit wird auch das Ändern der Parameter unserer Simulation zur Laufzeit unterstützt,
wobei insbesondere auf ein gleichbleibendes Leistungsprofil geachtet wird. Dies ermöglicht zum Beispiel Designern
unterschiedlichste Umgebungsparameter anzupassen und direkt visuelles Feedback zu erhalten, oder innerhalb von Sekunden
(aber dennoch mit fließendem Übergang) einen strahlend blauem Himmel in die Kulisse eines Unwetters zu verwandeln.

\clearpage
