\documentclass[german]{beamer}
\usepackage{mathtools}
\usepackage{babel}

\usepackage{csquotes}
\usepackage{float}

\usepackage{tikz}
\usetikzlibrary{intersections, positioning, calc, fit}

\usepackage{graphicx}
\graphicspath{{./media/}}

\usepackage{pgfplots}
\usepgfplotslibrary{polar}

\title{Dynamisches Himmel-Rendering}
\author{Simon Katz}
\institute{Rheinische Friedrich-Wilhelms-Universität Bonn}
\date{02.03.2022}

% from https://tex.stackexchange.com/questions/66216/draw-arc-in-tikz-when-center-of-circle-is-specified
\def\centerarc[#1](#2)(#3:#4:#5);%
%Syntax: [draw options] (center) (initial angle:final angle:radius)
    {
    %\draw[#1] ($(#2)+({#5*cos(#3)},{#5*sin(#3)})$) arc (#3:#4:#5);
    \draw[#1]([shift=(#3:#5)]#2) arc (#3:#4:#5);
    }

% from https://tex.stackexchange.com/questions/354746/pgfplots-with-function-definition-and-then-use-it-by-replacing-a-constant
\pgfplotsset{
	compat=1.12,
	/pgf/declare function={
		F_M(\t,\g) = ((3*(1-\g^2))/(3*(2+2*\g^2))*(1+cos(\t)^2)/(1+\g^2-2*\g*cos(\t))^1.5);
		F_R(\t) = (8/10)*(7/5+1/2*cos(\t)^2)
	}
}


\begin{document}

\frame{\titlepage}

\begin{frame}
\frametitle{Inhalt}
\tableofcontents
\end{frame}

\section{Grundlagen}

\begin{frame}
\frametitle{Grundlagen: Atmosphäre}

\begin{itemize}
	\item Atmosphäre besteht aus Partikeln (Moleküle und Aerosole)
	\item Licht wird abhängig von Wellenlänge gestreut
	\item Streuung sowohl ins Auge, als auch aus dem Pfad zum Auge heraus
	\item Absorption durch Aerosole
\end{itemize}

\begin{figure}[h]
	\centering
	\begin{tikzpicture}
	\useasboundingbox (0.0) rectangle (8.4,4.9);
	\begin{scope}[transform canvas={scale=0.7}]
	\clip (0,0) rectangle (12,7);
	\filldraw[fill=blue!20] (1, -10) circle (16);
	\filldraw[fill=green!20] (1, -10) circle (11);

	\path[name path=atmosphere] (1, -10) circle (16);

	\coordinate (eye) at (1, 2);
	\centerarc[thick](eye)(20:-10:.4);
	\draw[] (eye) -- ($(eye) + .5*({cos(20)}, {sin(20)})$);
	\draw[] (eye) -- ($(eye) + .5*({cos(-10)}, {sin(-10)})$);
	\path[name path=view] (eye) -- ($(eye) + 10*({cos(5)}, {sin(5)})$);

	\path[name path=eye_c1] (eye) circle (.55);

	\path[name path=eye_c2] (eye) circle (1.7);
	\path[name path=eye_forw1] (eye) -- ($(eye) + 10*({cos(13)}, {sin(13)})$);
	\node[name intersections={of=eye_c1 and eye_forw1}] (eye_c1_i1) at (intersection-1) {};
	\node[name intersections={of=eye_c2 and eye_forw1}] (eye_c2_i1) at (intersection-1) {};

	\path[name path=eye_c3] (eye) circle (3.5);
	\path[name path=eye_forw2] (eye) -- ($(eye) + 10*({cos(0)}, {sin(0)})$);
	\node[name intersections={of=eye_c1 and eye_forw2}] (eye_c1_i2) at (intersection-1) {};
	\node[name intersections={of=eye_c3 and eye_forw2}] (eye_c3_i1) at (intersection-1) {};

	\path[name path=eye_c4] (eye) circle (5);
	\path[name path=eye_forw3] (eye) -- ($(eye) + 10*({cos(-9)}, {sin(-9)})$);
	\node[name intersections={of=eye_c1 and eye_forw3}] (eye_c1_i3) at (intersection-1) {};
	\node[name intersections={of=eye_c4 and eye_forw3}] (eye_c4_i1) at (intersection-1) {};

	% \coordinate(pb) at (intersection 1 of atmosphere and view);

	\coordinate (p1) at (3.8, 4);
	\path[name path=pp1] (p1) -- ($(p1) + 10*({cos(40)}, {sin(40)})$);
	\draw[->, dashed, name intersections={of=atmosphere and pp1}]
		(intersection-1) --
		(p1) --
		(eye_c2_i1.center) --
		(eye_c1_i1.center);

	\path[name path=pp2] (eye_c3_i1.center) -- ($(eye_c3_i1.center) + 10*({cos(40)}, {sin(40)})$);
	\draw[->, dashed, name intersections={of=atmosphere and pp2}]
		(intersection-1) --
		(eye_c3_i1.center) --
		(eye_c1_i2.center);

	\coordinate (out_mid) at ($(eye_c4_i1.center)!.8!(eye_c1_i3.center)$);
	\path[name path=pp3] (eye_c4_i1.center) -- ($(eye_c4_i1.center) + 10*({cos(40)}, {sin(40)})$);
	\draw[->, dashed, name intersections={of=atmosphere and pp3}]
		(intersection-1) --
		(eye_c4_i1.center) --
		(out_mid) -- (.5, 1.6);

	\coordinate (l2) at ($(1, -10) + 16.5*({cos(73)}, {sin(73)})$);
	\draw[thick, <-, yellow!80!orange] (l2) -- ++($({cos(40)}, {sin(40)})$);
	
	\coordinate (l1) at ($(1, -10) + 16.5*({cos(69)}, {sin(69)})$);
	\draw[thick, <-, yellow!80!orange] (l1) -- ++($({cos(40)}, {sin(40)})$);
	
	\coordinate (l3) at ($(1, -10) + 16.5*({cos(64)}, {sin(64)})$);
	\draw[thick, <-, yellow!80!orange] (l3) -- ++($({cos(40)}, {sin(40)})$);
	
	\coordinate (l4) at ($(1, -10) + 16.5*({cos(59)}, {sin(59)})$);
	\draw[thick, <-, yellow!80!orange] (l4) -- ++($({cos(40)}, {sin(40)})$);
	
	\coordinate (l5) at ($(1, -10) + 16.5*({cos(56)}, {sin(56)})$);
	\draw[thick, <-, yellow!80!orange] (l5) -- ++($({cos(40)}, {sin(40)})$);

	\end{scope}
	\end{tikzpicture}
	% \caption{Mehrfache-, Einfache- und Ausstreuung}
	\label{streuung}
\end{figure}

\end{frame}

\begin{frame}
\frametitle{Grundlagen: Aerial Perspective}

\begin{itemize}
	\item Aus der Atmosphäre gestreutes Licht attenuiert Farbe weit entfernter Objekte
\end{itemize}

\begin{figure}[h]
	{\centering
	\begin{tikzpicture}
	\useasboundingbox (0.0) rectangle (8.4,4.9);
	\begin{scope}[transform canvas={scale=0.7}]
	\fill[fill=blue!20] (0,0) rectangle (12,7);
	\fill[fill=green!20] (0,0) rectangle (12,1);

	\coordinate (eye) at (1, 2);
	\centerarc[thick](eye)(20:-10:.4);
	\draw[] (eye) -- ($(eye) + .5*({cos(20)}, {sin(20)})$);
	\draw[] (eye) -- ($(eye) + .5*({cos(-10)}, {sin(-10)})$);
	\coordinate (eye_to) at ($(eye) + 10*({cos(10)}, {sin(10)})$);
	\path[name path=view] (eye) -- (eye_to);

	\path[name path=tri, fill=red] (8,1) -- (9.5,5) -- (11,1) -- cycle;

	\draw[dashed, red] ($(eye)!0.79!(eye_to)$) -- ($(eye)!0.70!(eye_to)$);
	\draw[dashed, orange] ($(eye)!0.71!(eye_to)$) -- ($(eye)!0.40!(eye_to)$);
	\draw[dashed, orange!70!blue] ($(eye)!0.395!(eye_to)$) -- ($(eye)!0.25!(eye_to)$);
	\draw[->, dashed, orange!60!blue] ($(eye)!0.25!(eye_to)$) -- ($(eye)!0.05!(eye_to)$);

	\coordinate (p1) at ($(eye)!0.70!(eye_to)$);
	\draw[->, dashed, yellow] ($(p1) + 2*({cos(55)}, {sin(55)})$) -- (p1);
	\node[below=1mm of p1] {$p_1$};

	\coordinate (p2) at ($(eye)!0.40!(eye_to)$);
	\draw[->, dashed, blue] ($(p2) + ({cos(320)}, {sin(320)})$) -- (p2);
	\node[below=1mm of p2] {$p_2$};

	\coordinate (p3) at ($(eye)!0.25!(eye_to)$);
	\draw[<-, dashed, orange!70!blue] ($(p3) + ({cos(170)}, {sin(170)})$) -- (p3);
	\node[below=1mm of p3] {$p_3$};

	\end{scope}
	\end{tikzpicture}
	}

	% \caption{Aerial Perspective}
	\label{AP}
\end{figure}

\end{frame}

\section{Einfluss der Atmosphäre}

\begin{frame}
\frametitle{Einfluss der Atmosphäre}

\includegraphics<1>[scale=0.12]{nogl_noap_nosun_noat.png}

\includegraphics<2>[scale=0.12]{nogl_noap_nosun.png}

\includegraphics<3>[scale=0.12]{nogl_noap.png}

\includegraphics<4>[scale=0.12]{nogl.png}

\includegraphics<5>[scale=0.12]{full.png}

\end{frame}

\section{Physikalische Umsetzung}

\begin{frame}
\frametitle{Streuung}

\begin{itemize}
	\item Alle Streuungen werden durch Mie- oder Rayleigh-Streuung modelliert
	\item Bei Partikeln mit $d \ll \frac{\lambda}{2\pi}$ (Lufmoleküle): Rayleigh, streut fast uniform
	\item Staub, Eis, Wassertropfen: Mie, starke Vorwärtsstreuung
	\item Rayleigh streut Wellenlängenabhängig, Mie nicht
\end{itemize}

\centering
\vspace{1.7em}
\includegraphics[scale=0.2]{scattering.png}

\end{frame}

\section{Mathematisches Modell}

\begin{frame}
\frametitle{Mathematisches Modell}

\begin{itemize}
\item Ziel: Die Intensität des Lichtes einer Wellenlänge $\lambda$, welches an einer Position $p$ mit einer
Blickrichtung $v$ einfällt.
\item Addiere Intensität für alle Streuungsordnungen
\item Vereinfachungen: Sonne als einzige Lichtquelle, die Strahlen dieser sind parallel
\end{itemize}

\end{frame}

\begin{frame}
\frametitle{Mathematisches Modell: Einzelne Streuung}

\begin{itemize}
	\item Modelliere zunächst eine einzelne Streuung
	\item Licht fällt bei $p$ aus $i$ ein und wird Richtung $o$ gestreut ($\theta = \angle(i,o)$)
	\item $\rho$: Dichtefunktion
	\item $F$: Phase function
	\item $\beta$: Streuungskoeffizient
\end{itemize}

\begin{equation*}
	S_{R,M}(p, i, o, \lambda) = \rho_{R,M}(p)F_{R,M}(\theta)\frac{\beta_{R,M}(\lambda)}{4\pi}
\end{equation*}
\begin{equation*}
	S(p, i, o, \lambda) = S_R(p,i,o,\lambda) + S_M(p,i,o,\lambda)
\end{equation*}

\end{frame}

\begin{frame}
\frametitle{Mathematisches Modell: Transmittance}

\begin{itemize}
	\item Transmittance modelliert Dämpfung durch die Atmosphäre zwischen zwei Punkten $p_a$, $p_b$
	\item $\beta^e$: extinction coefficient, modelliert Ausstreuung (Rayleigh und Mie) und Absorption (Mie und Ozon)
\end{itemize}

\begin{equation*}
	T(p_a, p_b, \lambda) = \exp\bigg(-\sum_{i\in R,M,O}^{} \beta^e_i(\lambda) \int_{p_a}^{p_b} \rho_i(p) \,dp\bigg)
\end{equation*}

\end{frame}

\begin{frame}
\frametitle{Mathematisches Modell: Einfachstreuung}

\begin{figure}[h]
	\centering
	\begin{tikzpicture}
	\useasboundingbox (0.0) rectangle (8.4,4.9);
	\begin{scope}[transform canvas={scale=0.7}]
	\clip (0,0) rectangle (12,7);
	\filldraw[fill=blue!20] (1, -10) circle (16);
	\filldraw[fill=green!20] (1, -10) circle (11);

	\path[name path=atmosphere] (1, -10) circle (16);

	\coordinate (eye) at (1, 2);
	\centerarc[thick](eye)(20:-10:.4);
	\draw[] (eye) -- ($(eye) + .5*({cos(20)}, {sin(20)})$);
	\draw[] (eye) -- ($(eye) + .5*({cos(-10)}, {sin(-10)})$);
	\path[name path=view] (eye) -- ($(eye) + 10*({cos(5)}, {sin(5)})$);

	% \coordinate(pb) at (intersection 1 of atmosphere and view);

	% \node[name intersections={of=atmospherea and view}] (intersection) {$p_b$};
	\node[red, name intersections={of=atmosphere and view}] (pb) at (intersection-1) {};
	\node[below=0 of pb.center] {$p_b$};
	\draw[<-, dashed] ($(eye) + .5*({cos(5)}, {sin(5)})$) -- (pb.center);

	\coordinate (p1) at ($(eye)!0.2!(pb)$);
	\node[below=0 of eye.center] {$p_a$};

	\path[name path=sun1] (p1) -- ($(p1) + 10*({cos(40)}, {sin(40)})$);
	\node[red, name intersections={of=atmosphere and sun1}] (pl1) at (intersection-1) {};
	\node[below=0 of p1.center] {$p_{1}$};
	\draw[dashed] (p1) -- (pl1.center);
	\node[below=1mm of pl1.center] {$p_{l1}$};

	\coordinate (p2) at ($(eye)!0.35!(pb)$);
	\path[name path=sun2] (p2) -- ($(p2) + 20*({cos(40)}, {sin(40)})$);
	\node[red, name intersections={of=atmosphere and sun2}] (pl2) at (intersection-1) {};
	\node[below=0 of p2.center] {$p_{2}$};
	\draw[dashed] (p2) -- (pl2.center);
	\node[below=1mm of pl2.center] {$p_{l2}$};

	\coordinate (p3) at ($(eye)!0.7!(pb)$);
	\path[name path=sun3] (p3) -- ($(p3) + 30*({cos(40)}, {sin(40)})$);
	\node[below=0 of p3.center] {$p_{3}$};
	\node[red, name intersections={of=atmosphere and sun3}] (pl3) at (intersection-1) {};
	\draw[dashed] (p3) -- (pl3.center);
	\node[below=1mm of pl3.center] {$p_{l3}$};

	\coordinate (l2) at ($(1, -10) + 16.5*({cos(73)}, {sin(73)})$);
	\draw[thick, <-, yellow!80!orange] (l2) -- ++($({cos(40)}, {sin(40)})$);

	\coordinate (l1) at ($(1, -10) + 16.5*({cos(69)}, {sin(69)})$);
	\draw[thick, <-, yellow!80!orange] (l1) -- ++($({cos(40)}, {sin(40)})$);

	\coordinate (l3) at ($(1, -10) + 16.5*({cos(64)}, {sin(64)})$);
	\draw[thick, <-, yellow!80!orange] (l3) -- ++($({cos(40)}, {sin(40)})$);

	\coordinate (l4) at ($(1, -10) + 16.5*({cos(59)}, {sin(59)})$);
	\draw[thick, <-, yellow!80!orange] (l4) -- ++($({cos(40)}, {sin(40)})$);

	\coordinate (l5) at ($(1, -10) + 16.5*({cos(56)}, {sin(56)})$);
	\draw[thick, <-, yellow!80!orange] (l5) -- ++($({cos(40)}, {sin(40)})$);

	\end{scope}
	\end{tikzpicture}
	\label{SS-image}
\end{figure}

\begin{equation*}
	I^{(1)}_S(p_a,v,l,\lambda) = \int_{p_a}^{p_b}  I(\lambda) T(p_l, p, \lambda)S(p, -l, -v, \lambda) T(p, p_a, \lambda) \,dp
\end{equation*}

\end{frame}

\begin{frame}
\frametitle{Mathematisches Modell: Mehrfachstreuung}

\begin{itemize}
	\item Iteratives Verfahren für Einfache Streuung $\rightarrow$ Mehrfache Streuung
	\item Zwei Schritte: Gathering und Mehrfache Streuung (Unterteilung um nah an der Implementierung zu bleiben)
\end{itemize}
\end{frame}

\begin{frame}
\frametitle{Mathematisches Modell: Gathering}

\begin{figure}
	\centering
	\begin{tikzpicture}
	\useasboundingbox (0.0) rectangle (8.4,4.9);
	\begin{scope}[transform canvas={scale=0.7}]
	\clip (0,0) rectangle (12,7);
	\filldraw[fill=blue!20] (1, -10) circle (16);
	\filldraw[fill=green!20] (1, -10) circle (11);

	\path[name path=atmosphere] (1, -10) circle (16);

	\coordinate (eye) at (1, 2);

	\node[circle, minimum size=.15cm, inner sep=0] (p2) at ($(eye)!0.35!(pb)$) {};
	\node[minimum size=1cm, circle, label=below:$p$] (c2) at (p2) {};
	\draw[dashed] (c2.97) -- (p2.center);
	\draw[dashed] (c2.25) -- (p2.center);
	\draw[dashed] (c2.169) -- (p2.center);
	\draw[dashed] (c2.241) -- (p2.center);
	\draw[dashed] (c2.313) -- (p2.center);

	\draw[->, dashed] (p2.center) -- ($(eye)!0.25!(pb)$);

	\coordinate (l2) at ($(1, -10) + 16.5*({cos(73)}, {sin(73)})$);
	\draw[thick, <-, yellow!80!orange] (l2) -- ++($({cos(40)}, {sin(40)})$);

	\coordinate (l1) at ($(1, -10) + 16.5*({cos(69)}, {sin(69)})$);
	\draw[thick, <-, yellow!80!orange] (l1) -- ++($({cos(40)}, {sin(40)})$);

	\coordinate (l3) at ($(1, -10) + 16.5*({cos(64)}, {sin(64)})$);
	\draw[thick, <-, yellow!80!orange] (l3) -- ++($({cos(40)}, {sin(40)})$);

	\coordinate (l4) at ($(1, -10) + 16.5*({cos(59)}, {sin(59)})$);
	\draw[thick, <-, yellow!80!orange] (l4) -- ++($({cos(40)}, {sin(40)})$);

	\coordinate (l5) at ($(1, -10) + 16.5*({cos(56)}, {sin(56)})$);
	\draw[thick, <-, yellow!80!orange] (l5) -- ++($({cos(40)}, {sin(40)})$);

	\end{scope}
	\end{tikzpicture}
	\label{MS-image}
\end{figure}

\begin{itemize}
	\item Berechnet Intensität des $k-1$-fach gestreuten Lichtes, das an einem Punkt $p$ in Richtung $o$ gestreut wird.
\end{itemize}

\begin{equation*}
	G^{(k)}(p,o,l,\lambda) = \int_{4\pi}^{} I^{(k-1)}(p,\omega,l,\lambda) S(p,-\omega,o,\lambda) \,d\omega
\end{equation*}
\end{frame}

\begin{frame}
\frametitle{Mathematisches Modell: Mehrfache Streuung}

\begin{figure}
	\centering
	\begin{tikzpicture}
	\useasboundingbox (0.0) rectangle (8.4,4.9);
	\begin{scope}[transform canvas={scale=0.7}]
	\clip (0,0) rectangle (12,7);
	\filldraw[fill=blue!20] (1, -10) circle (16);
	\filldraw[fill=green!20] (1, -10) circle (11);

	\path[name path=atmosphere] (1, -10) circle (16);

	\coordinate (eye) at (1, 2);
	\centerarc[thick](eye)(20:-10:.4);
	\draw[] (eye) -- ($(eye) + .5*({cos(20)}, {sin(20)})$);
	\draw[] (eye) -- ($(eye) + .5*({cos(-10)}, {sin(-10)})$);
	\path[name path=view] (eye) -- ($(eye) + 10*({cos(5)}, {sin(5)})$);

	% \coordinate(pb) at (intersection 1 of atmosphere and view);

	% \node[name intersections={of=atmospherea and view}] (intersection) {$p_b$};
	\node[red, name intersections={of=atmosphere and view}] (pb) at (intersection-1) {};
	\node[below=0 of pb.center] {$p_b$};
	\draw[dashed, <-] ($(eye) + .5*({cos(5)}, {sin(5)})$) -- (pb.center);
	\node[below=0 of eye.center] {$p_a$};

	\node[circle, minimum size=.15cm, inner sep=0] (p1) at ($(eye)!0.2!(pb)$) {};
	\node[minimum size=1cm, circle, label=below:$p_1$] (c1) at (p1) {};
	\draw[->, dashed] (c1.72) -- (p1.72);
	\draw[->, dashed] (c1.0) -- (p1.0);
	\draw[->, dashed] (c1.144) -- (p1.144);
	\draw[->, dashed] (c1.216) -- (p1.216);
	\draw[->, dashed] (c1.288) -- (p1.288);

	\node[circle, minimum size=.15cm, inner sep=0] (p2) at ($(eye)!0.35!(pb)$) {};
	\node[minimum size=1cm, circle, label=below:$p_2$] (c2) at (p2) {};
	\draw[->, dashed] (c2.97) -- (p2.97);
	\draw[->, dashed] (c2.25) -- (p2.25);
	\draw[->, dashed] (c2.169) -- (p2.169);
	\draw[->, dashed] (c2.241) -- (p2.241);
	\draw[->, dashed] (c2.313) -- (p2.313);

	\node[circle, minimum size=.15cm, inner sep=0] (p3) at ($(eye)!0.7!(pb)$) {};
	\node[minimum size=1cm, circle, label=below:$p_3$] (c3) at (p3) {};
	\draw[->, dashed] (c3.121) -- (p3.121);
	\draw[->, dashed] (c3.49) -- (p3.49);
	\draw[->, dashed] (c3.193) -- (p3.193);
	\draw[->, dashed] (c3.265) -- (p3.265);
	\draw[->, dashed] (c3.337) -- (p3.337);

	\coordinate (l2) at ($(1, -10) + 16.5*({cos(73)}, {sin(73)})$);
	\draw[thick, <-, yellow!80!orange] (l2) -- ++($({cos(40)}, {sin(40)})$);

	\coordinate (l1) at ($(1, -10) + 16.5*({cos(69)}, {sin(69)})$);
	\draw[thick, <-, yellow!80!orange] (l1) -- ++($({cos(40)}, {sin(40)})$);

	\coordinate (l3) at ($(1, -10) + 16.5*({cos(64)}, {sin(64)})$);
	\draw[thick, <-, yellow!80!orange] (l3) -- ++($({cos(40)}, {sin(40)})$);

	\coordinate (l4) at ($(1, -10) + 16.5*({cos(59)}, {sin(59)})$);
	\draw[thick, <-, yellow!80!orange] (l4) -- ++($({cos(40)}, {sin(40)})$);

	\coordinate (l5) at ($(1, -10) + 16.5*({cos(56)}, {sin(56)})$);
	\draw[thick, <-, yellow!80!orange] (l5) -- ++($({cos(40)}, {sin(40)})$);

	\end{scope}
	\end{tikzpicture}
	\label{MS-image}
\end{figure}

\begin{equation*}
	I_S^{(k)}(p,v,l,\lambda) = \int_{p_a}^{p_b} G^{(k)}(p,-v, l, \lambda) T(p,p_a) \,dp
\end{equation*}

\begin{equation*}
	I_S = \sum^{K}_{i=1} I_S^{(i)}
\end{equation*}
\end{frame}

\section{Implementierung}

\begin{frame}
\frametitle{Implementierung: Vereinfachungen}

\begin{itemize}
	\item Ziel Echtzeitrendering $\rightarrow$ Vorberechnen
	\item Annahmen um Dimensionen der Texturen zu verringern+Vorberechnung zu vereinfachen:
		\begin{itemize}
			\item Die Erde ist eine perfekte Kugel und $\rho_{R,M,O}(p)$ hängen nur von der Höhe von p ab:\\ 3
			$\rightarrow$ 1 Dimension für $p$
			\item Die Azimuth-Differenz zwischen Blick- und Sonnenwinkel wird während der Vorberechnung ignoriert und
			mit einer modifizierten $F_R(\theta)$ zur Laufzeit ausgewertet:\\ Blickrichtung 2 $\rightarrow$ 1 Dimension
			(Zenithwinkel), Sonnenrichtung 2 $\rightarrow$ 1 Dimension (Zenithwinkel)
		\end{itemize}
\end{itemize}
\end{frame}

\begin{frame}
\frametitle{Implementierung: $F_{R,M}(\theta)$}

\begin{itemize}
	\item $F_R$ (Rayleigh Phase Function): $F_R(\theta) = \frac{8}{10}(\frac{7}{5}+\frac{1}{2}\cos^2(\theta))$. Streut fast
	uniform
	\item $F_M$ (Mie Phase Function, hier modifizierte Henyey-Greenstein Funktion): $F_M(\theta) =
	\frac{3(1-g^2)}{3(2+2g^2)} \frac{(1+\cos^2(\theta))}{(1+g^2-2g\cos(\theta))^{\frac{3}{2}}}$, hat eine starke, von
	$g$ bestimmte, Vorwärtsstreuung. Abgebildet mit $g=0.2,0.4,0.6$
\end{itemize}

\begin{tabular}{c c}
\begin{tikzpicture}
	\begin{polaraxis}[
		xmin=0, xmax=180,
		width=5cm,
		ytick style={draw=none},
		yticklabel style={
			anchor=north,
			yshift=-0.3*\pgfkeysvalueof{/pgfplots/major tick length}
		},
		% legend style={
		% 	at={(.1,.1)},
		% 	anchor=north east
		% }
	]
	\addplot+[mark=none,domain=0:360,samples=600] {F_M(x, .2)};
	\addplot+[mark=none,domain=0:360,samples=600] {F_M(x, .4)};
	\addplot+[mark=none,domain=0:360,samples=600] {F_M(x, .6)};
	% \legend{$g=0.2$, $g=0.4$, $g=0.6$}
	\end{polaraxis}
\end{tikzpicture}
 & 
\begin{tikzpicture}
	\begin{polaraxis}[
		xmin=0, xmax=180,
		width=5cm,
		ytick style={draw=none},
		yticklabel style={
			anchor=north,
			yshift=-0.3*\pgfkeysvalueof{/pgfplots/major tick length}
		}
	]
	\addplot+[mark=none,domain=0:360,samples=600] {F_R(x)};
	\addplot+[mark=none,domain=0:360,samples=600] {F_R(x)};
	\addplot+[mark=none,domain=0:360,samples=600] {F_R(x)};
	\end{polaraxis}
\end{tikzpicture}
\end{tabular}

\end{frame}

\begin{frame}
\frametitle{Implementierung: $\beta_{R,M}(\lambda)$}

\begin{itemize}
	\item Rayleigh: kann mit $\beta_R(\lambda) = 8\pi^3 \frac{(n^2-1)^2}{3N_e\lambda^4}$ bestimmt werden ($N_e$:
	molekulare Dichte auf Meeresspiegel, $n$: refraktiver Index der Luft). Hier $\beta_{R_\text{RGB}}=(6.55e^{-6}, 1.73e^{-5},
	2.30e^{-5})$ für $\lambda_\text{RGB}=(650,510,475)\text{nm}$, $N_e = 2.545e^{25}$, $n = 1.0003$.
	\item Mie: kompliziertere Berechnung, hier einfach $\beta_M=(2e^{-6},2e^{-6},2e^{-6})$ (Wellenlängenunabhängig!)
\end{itemize}

\end{frame}

\begin{frame}
\frametitle{Implementierung: Umsetzung}

\begin{itemize}
	\item Streuung wird für die RGB entsprechenden Wellenlängen (hier: 650,510,475nm) bestimmt
	\item Atmosphäre wird auf Far Plane gezeichnet
\end{itemize}

\end{frame}


\end{document}
